\href{https://travis-ci.org/bblanchon/ArduinoJson}{\tt !\mbox{[}Build Status\mbox{]}(https\+://travis-\/ci.\+org/bblanchon/\+Arduino\+Json.\+svg?branch=master)} \href{https://coveralls.io/r/bblanchon/ArduinoJson?branch=master}{\tt !\mbox{[}Coverage Status\mbox{]}(https\+://img.\+shields.\+io/coveralls/bblanchon/\+Arduino\+Json.\+svg)}

{\itshape An elegant and efficient J\+S\+O\+N library for embedded systems.}

It\textquotesingle{}s design to have the most intuitive A\+P\+I, the smallest footprint and works without any allocation on the heap (no malloc).

It has been written with Arduino in mind, but it isn\textquotesingle{}t linked to Arduino libraries so you can use this library in any other C++ project.

\subsection*{Features }


\begin{DoxyItemize}
\item J\+S\+O\+N decoding
\item J\+S\+O\+N encoding (with optional indentation)
\item Elegant A\+P\+I, very easy to use
\item Fixed memory allocation (no malloc)
\item Small footprint
\item M\+I\+T License
\end{DoxyItemize}

\subsection*{Quick start }

\paragraph*{Decoding / Parsing}

\begin{DoxyVerb}char json[] = "{\"sensor\":\"gps\",\"time\":1351824120,\"data\":[48.756080,2.302038]}";

StaticJsonBuffer<200> jsonBuffer;

JsonObject& root = jsonBuffer.parseObject(json);

const char* sensor = root["sensor"];
long time          = root["time"];
double latitude    = root["data"][0];
double longitude   = root["data"][1];
\end{DoxyVerb}


\paragraph*{Encoding / Generating}

\begin{DoxyVerb}StaticJsonBuffer<200> jsonBuffer;

JsonObject& root = jsonBuffer.createObject();
root["sensor"] = "gps";
root["time"] = 1351824120;

JsonArray& data = root.createNestedArray("data");
data.add(48.756080, 6);  // 6 is the number of decimals to print
data.add(2.302038, 6);   // if not specified, 2 digits are printed

root.printTo(Serial);
// This prints:
// {"sensor":"gps","time":1351824120,"data":[48.756080,2.302038]}
\end{DoxyVerb}


\subsection*{Documentation }

The documentation is available online in the \href{https://github.com/bblanchon/ArduinoJson/wiki}{\tt Arduino J\+S\+O\+N wiki}

\subsection*{Testimonials }

From Arduino\textquotesingle{}s Forum user {\ttfamily jflaplante}\+: \begin{quote}
I tried a\+Json json-\/arduino before trying your library. I always ran into memory problem after a while. I have no such problem so far with your library. It is working perfectly with my web services. \end{quote}


From Arduino\textquotesingle{}s Forum user {\ttfamily gbathree}\+: \begin{quote}
Thanks so much -\/ this is an awesome library! If you want to see what we\textquotesingle{}re doing with it -\/ the project is located at www.\+photosynq.\+org. \end{quote}


From Stack\+Overflow user {\ttfamily thegreendroid}\+: \begin{quote}
It has a really elegant, simple A\+P\+I and it works like a charm on embedded and Windows/\+Linux platforms. We recently started using this on an embedded project and I can vouch for its quality. \end{quote}


From Git\+Hub user {\ttfamily zacsketches}\+:

\begin{quote}
Thanks for a great library!!! I\textquotesingle{}ve been watching you consistently develop this library over the past six months, and I used it today for a publish and subscribe architecture designed to help hobbyists move into more advanced robotics. Your library allowed me to implement remote subscription in order to facilitate multi-\/processor robots. Arduino\+Json saved me a week\textquotesingle{}s worth of time!! \end{quote}




Found this library useful? \href{https://www.paypal.com/cgi-bin/webscr?cmd=_donations&business=donate%40benoitblanchon%2efr&lc=GB&item_name=Benoit%20Blanchon&item_number=Arduino%20JSON&currency_code=EUR&bn=PP%2dDonationsBF%3abtn_donate_LG%2egif%3aNonHosted}{\tt Help me back with a donation!} \+:smile\+: 